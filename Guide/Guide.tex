\documentclass[letter]{article}

%% Language and font encodings
\usepackage[english]{babel}
\usepackage[utf8x]{inputenc}
\usepackage[T1]{fontenc}

%% Sets page size and margins
\usepackage[a4paper,top=3cm,bottom=2cm,left=3cm,right=3cm,marginparwidth=1.75cm]{geometry}

%% Useful packages
\usepackage{amsmath}
\usepackage{graphicx}
\usepackage[colorinlistoftodos]{todonotes}
\usepackage[colorlinks=true, allcolors=blue]{hyperref}

\title{VR Build guide v0.1}
\author{Noah Pettit}

\begin{document}
\maketitle

%%\begin{abstract}
%%Your abstract.
%%\end{abstract}

\section{Overview}

For each section, there are two main subsections: "part preparation" and "assembly". Part preparation will include a full parts list and describe (when applicable, how to make or modify parts before beginning assembly). 

\subsection{Main components}
\begin{enumerate}
    \item Back projection screen
    \item Ball cup assembly
    \item Behavior electronics
    \item Enclosure
    \item Reward delivery system
    
\end{enumerate}

\section{Back projection screen}

The back projection screen assembly is comprised of 4 pieces of laser cut acrylic that together house the two mirrors, a parabolic projector screen, and mounting points for the projector. See figure \ref{fig:bp_overview} for an overview of this assembly.
The pieces of acrylic are joined and mounted on vertical 80-20 (25mm) rails. The entire housing should eventually be bolted down to an air-table or breadboard once the final position is set. 

Equipment you need access to: 

laser cutter (power to cut through 1/4" acrylic, min bed size 12 x 24")
Chop saw (to cut 80-20 rails, optional, thorlabs rails)
1/4-20 thread cutting tap and power drill



\subsection{Part preparation}
For this assembly you will need:
\begin{table}
\centering
\begin{tabular}{l|r}
Quantity & item \\\hline
Side pieces & 2 \\
Laser cut top & 1 \\
Laser cut bottom & 1 \\

\end{tabular}
\caption{\label{tab:widgets}An example table.}
\end{table}

\begin{enumerate}
    \item Laser cut the parts. 
    \item Tap holes
    \item Cut screen film to size
    \item Join the two sides using 15" rails
    \item Join the top and the bottom to the sides using the 1" angle brackets
    \item Place the assembly top side down and attach the 4 legs using T-nuts and low profile hex screws
    
\end{enumerate}

\subsection{Required parts}




\begin{table}
\centering
\begin{tabular}{l|r}
Item & Quantity \\\hline
Side pieces & 2 \\
Laser cut top & 1 \\
Laser cut bottom & 1 \\

\end{tabular}
\caption{\label{tab:widgets}An example table.}
\end{table}

\section{Behavior electronics}


\section{Ball cup assembly}

\section{FAQs}
\subsection{What are the computer hardware requirements?}
The projector is driven simply as another monitor, so this really depends on software that you are using. Consult the documentation for whatever visual stimulus or VR system you are using. For example, psychtoolbox, psychopy, and the virmen documentation have sections on this. Our computers use GTX 1060 cards (https://www.amazon.com/GR8-II-T044Z-Desktop-i5-7400-GeForce/dp/B06WLHT15C is one model we have used successfully) but this is probably overkill for most of the worlds we use. Integrated graphics are a bad idea and will likely result in <60FPS. For ViRMEn we suggest at least a 3.00GHz CPU and 8GB of RAM. 

\subsection{What are the pros and cons of using this setup over multiple monitors?}
The pros (+) and cons (-) of multiple monitors:

Multiple monitors:

+ slightly better availability and choice of different screen types and resolutions. may be better if you care about precise visual stimulus timing as gaming monitors are available w/ high frame rates and low input lag.

+ Potential to be more compact (although this design is already very compact).

- Requires either better graphics card to drive 3+ monitors or additional hardware such as "triple head-to-go" (which may complicate transformation functions and introduce lag - you'd need to test). 

- Bezels: not a big deal for navigation stuff, but does disrupt things like whole-field optic flow, visual RF mapping in same experiment, etc. May be less immersive.

- May be harder to correct for illumination across visual field as that can vary w/ angle that the mouse is seeing the screen from.

- Less flexible configuration - with our design you can cut holes, notches, slots into any part of the screen etc to fit optics, microscopes, cameras, etc. You can easily resize the screen and adjust design. We have found this to be very useful.

We have used both a 3 monitor setup and this design and have not found a difference in terms of behavior.

\subsection{How to include Figures}

%\begin{figure}
%\centering
%\includegraphics[width=1\textwidth]{techDWG.PNG}
%\caption{\label{fig:bp_overview}This frog was uploaded via the project menu.}
%\end{figure}

\subsection{How to add Comments}

Comments can be added to your project by clicking on the comment icon in the toolbar above. % * <john.hammersley@gmail.com> 2016-07-03T09:54:16.211Z:
%
% Here's an example comment!
%
To reply to a comment, simply click the reply button in the lower right corner of the comment, and you can close them when you're done.

Comments can also be added to the margins of the compiled PDF using the todo command\todo{Here's a comment in the margin!}, as shown in the example on the right. You can also add inline comments:

\todo[inline, color=green!40]{This is an inline comment.}

\subsection{How to add Tables}

Use the table and tabular commands for basic tables --- see Table~\ref{tab:widgets}, for example. 

\begin{table}
\centering
\begin{tabular}{l|r}
Item & Quantity \\\hline
Widgets & 42 \\
Gadgets & 13
\end{tabular}
\caption{\label{tab:widgets}An example table.}
\end{table}

\subsection{How to write Mathematics}

\LaTeX{} is great at typesetting mathematics. Let $X_1, X_2, \ldots, X_n$ be a sequence of independent and identically distributed random variables with $\text{E}[X_i] = \mu$ and $\text{Var}[X_i] = \sigma^2 < \infty$, and let
\[S_n = \frac{X_1 + X_2 + \cdots + X_n}{n}
      = \frac{1}{n}\sum_{i}^{n} X_i\]
denote their mean. Then as $n$ approaches infinity, the random variables $\sqrt{n}(S_n - \mu)$ converge in distribution to a normal $\mathcal{N}(0, \sigma^2)$.
% * <noahpettit@gmail.com> 2017-02-08T03:38:17.813Z:
% 
% testing comment 
% 
% ^.

\subsection{How to create Sections and Subsections}

Use section and subsections to organize your document. Simply use the section and subsection buttons in the toolbar to create them, and we'll handle all the formatting and numbering automatically.

\subsection{How to add Lists}

You can make lists with automatic numbering \dots

\begin{enumerate}
\item Like this,
\item and like this.
\end{enumerate}
\dots or bullet points \dots
\begin{itemize}
\item Like this,
\item and like this.
\end{itemize}

\end{document}